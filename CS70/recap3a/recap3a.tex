\documentclass[11 pt]{scrartcl}
\usepackage[header, margin, koma]{tyler}

\newcommand{\hwtitle}{Discussion 3A Recap}

\pagestyle{fancy}
\fancyhf{}
\fancyhead[l]{\hwtitle{}}
\fancyhead[r]{Tyler Zhu}
\cfoot{\thepage}

\begin{document} 
\title{\Large \hwtitle{}}
\author{\large Tyler Zhu}
\date{\large\today}

\maketitle 

\section{Modular Arithmetic}
\begin{definition}
    We say that $a$ is \emph{congruent} to $b \pmod{m}$ if 
    \[ a\equiv b \pmod{m} \iff m | a-b \iff a-b = m\cdot k, \quad k\in\ZZ .\] 
\end{definition}

One interpretation is that mod $m$ gets the remainder when we divide by $m$, but the mod operator is more powerful than just that. For example, we have that 
\begin{align*}
    \dots \equiv -9 \equiv -4 \equiv \boxed{1} \equiv 6 \equiv 11 \equiv \dots \pmod{5} \\ 
    \dots \equiv -8 \equiv -3 \equiv \boxed{2} \equiv 7 \equiv 12 \equiv \dots \pmod{5} \\ 
    \dots \equiv -7 \equiv -2 \equiv \boxed{3} \equiv 8 \equiv 13 \equiv \dots \pmod{5} 
\end{align*}

Given that so many numbers are equivalent to each other when working over a certain modulus, it helps us to agree upon a set of \emph{representatives} for each equivalence class of numbers. In the above example, the representatives for each class has been boxed. In general, the representatives are $\{0, 1, \dots, m-1\}$. 

To drive this point home, compare to how we say that all of the following fractions are the same, but we use the boxed one as their representative (namely $\frac 13$): 
\[ \dots = \dfrac{-3}{-9} = \dfrac{-2}{-6} = \dfrac{-1}{-3} = \boxed{\dfrac{1}{3}} = \dfrac{2}{6} = \dfrac{3}{9} = \dots \]

Doing math over a modulus is similar to normal arithmetic; addition, subtraction, multiplication, and exponentiation all hold. 

Division is tricky however. Being able to divide by $a$ is equivalent to having an \emph{inverse}, i.e. a number $x$ which makes $ax \equiv 1 \pmod{m}$. The existence of an inverse is equivalent to having a solution $(x,k)$ over integers to the equation $ax = 1 + m\cdot k$. We saw in the notes (and you can reason why) that this happens only when $\gcd(a,m) = 1$, and is unique mod $m$. 

In general, it's a good question to ask when we have solutions to the equation 
\[ ax + by = c\] 
for $a,b,d \in \ZZ$. If we let $d = \gcd(a,b)$, then solutions exist only when $d | c$ (take this equation mod $d$ if you don't believe me). Additionally, mod $a$ or $b$ these solutions are unique. 

We even have an algorithm called the \emph{Extended Euclidean Algorithm} which helps us find solutions to $ax+by = d$, from which we can get solutions to any equation in the above form (the Euclidean Algorithm lets us compute $\gcd(a,b)$ efficiently; you can imagine why extending it lets us recover the above solutons). 

One common followup is finding the number of solutions to an equation like $10x \equiv 25 \pmod{30}$. Think about this (you may find the above context helpful).

\section{Tips}
\itemnum
    \ii ``Find the last digit'' $\rar$ Take mod $10$. ``Last two digits'' $\rar$ Take mod $100$, and so on. 
    \ii It's useful to write a number $n$ as $n = \sum_{i=0}^k d_i 10^i = \overline{d_kd_{k-1}\dots d_1d_0}$ when taking mods. 
    \ii Recall that $a\equiv b \implies a^n \equiv b^n \pmod{n}$; in other words, reduce the bases of your powers. 
    \ii It can be helpful to work with different definitions of modular arithmetic. For example, showing that $3x\equiv 10 \pmod{21}$ has no solutions is easiest by demonstrating that $3x = 10 + 21k$ reduces to $0\equiv 1 \pmod{3}$.
    \ii While the EEA is great, finding inverses will often be faster/easier by writing out multiples of $m$. For example, if I'm finding the inverse of $9$ mod $11$, it's easier to look for a multiple of 9 in $1, 12, 23, 34, 45, \dots$ and then divide than to do the EEA. 
\itemend

\section{Extra Practice}
\begin{exercise}
    Prove that for any number $n$, the alternating sum of the digits of $n$, i.e. $d_0 - d_1 + d_2 - \dots$, is divisible by 11 if and only if $n$ is divisible by 11. 
\end{exercise}

\end{document}
